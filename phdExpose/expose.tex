\documentclass[11pt,a4paper]{article}

% Required packages
\usepackage[utf8]{inputenc}
\usepackage[T1]{fontenc}
\usepackage{times}
\usepackage[left=2.5cm,right=2.5cm,top=2.5cm,bottom=2.5cm]{geometry}
\usepackage{setspace}
\usepackage{hyperref}
\usepackage{natbib}
\usepackage{amsmath}
\usepackage{booktabs}
\usepackage{enumitem}
\usepackage{xcolor}
\usepackage{titlesec}
\usepackage{fancyhdr}

% Document settings
\onehalfspacing
\setlength{\parindent}{0pt}
\setlength{\parskip}{1em}

% Section spacing
\titlespacing*{\section}{0pt}{3.5ex plus 1ex minus .2ex}{2.3ex plus .2ex}
\titlespacing*{\subsection}{0pt}{3.25ex plus 1ex minus .2ex}{1.5ex plus .2ex}

% Headers and footers
\pagestyle{fancy}
\fancyhead{}
\fancyhead[R]{\thepage}
\fancyhead[L]{Doctoral dissertation exposé}
\fancyfoot{}
\fancyfoot[C]{\thepage}

% Title information
\title{\Large Doctoral dissertation exposé:\\[1em] 
       \huge\textbf{Fiscal policy in the euro area:\\
       Modeling, data quality, and policy implementation}}
\author{Krzysztof Bańkowski\\[0.5em]
        Student ID: 3005 70 007 964\\
        Graduate School of Economics, Finance, and Management (GSEFM)\\
        Goethe University Frankfurt}
\date{\today}

\begin{document}

\maketitle

\begin{abstract}
This dissertation explores the multifaceted aspects of fiscal policy in the euro area through four interconnected papers. The research combines theoretical modelling, empirical analysis, and policy evaluation to address crucial questions in fiscal policy effectiveness, data quality, and macroeconomic implications of fiscal rules and investment programmes. Using various methodological approaches, including DSGE modelling, semi-structural modelling, and real-time data analysis, the dissertation aims to contribute to both academic understanding and practical policy implementation.
\end{abstract}

\section*{Research topic and motivation}
This dissertation explores the intricacies of macroeconomic modelling of fiscal policy, with a particular focus on the euro area. The research is motivated by the critical role of fiscal policy as a macroeconomic instrument emerging since the Great Financial Crisis. The timing is particularly relevant as the EU just underwent an overhaul of its fiscal framework and going forward it faces substantial investment requirements to address the pressing needs of decarbonization, digitalization, and defence. This research addresses important gaps in our understanding of fiscal policy effectiveness, data quality, and the macroeconomic implications of fiscal rules and investment programmes. The projects will consider the wealth of macroeconomic models collected in the MMB database and its functionality to robustify the findings.

\section*{Research questions}
\begin{enumerate}[label=\textbf{RQ\arabic*.}, wide=0pt]
    \item How can semi-structural models, which combine sound theoretical foundations and data fit, be used for gauging potency of various fiscal policy instruments?
    \item To what extent are euro area quarterly fiscal data in real time reliable and predictable, and how do they compare to other macroeconomic data?
    \item What would have been the macroeconomic outcomes in the euro area if there had been full compliance with the Stability and Growth Pact, and what lessons can be drawn for the new EU economic governance framework?
    \item What are the comprehensive macroeconomic effects of implementing a large-scale EU investment package, particularly considering the financing structure?
\end{enumerate}

\section*{Research objectives}
The dissertation aims to take stock of fiscal policy modelling together with policy relevant applications in the euro area context. Specifically, it seeks to:
\begin{itemize}[wide=0pt]
    \item Develop and document a robust fiscal block within the ECB-BASE model for fiscal policy analysis
    \item Establish the reliability characteristics of quarterly fiscal data through systematic analysis of revisions
    \item Evaluate the counterfactual impact of full compliance with fiscal rules as formulated in the recently overhauled SGP
    \item Assess the macroeconomic implications of large-scale EU investment initiatives together with their fiscal cost
\end{itemize}
Through this multi-faceted approach, the research strives to bridge the gap between theoretical modelling and practical policy implementation.

\section*{Dissertation papers}

\subsection*{Paper 1 (single-authored): Fiscal policy in the semi-structural model ECB-BASE}
\textbf{Abstract:} Fiscal policy constitutes a key tool for business cycle stabilization next to monetary policy. In this context, having a well-suited macroeconomic model for analysing fiscal policy at an economic policy institution is of primary importance. This paper documents the fiscal block of the ECB-BASE, which is a semi-structural model for the euro area developed at the ECB for projections and policy analysis. The set-up of the fiscal block ensures comprehensive coverage of the government sector and tight links to the quarterly fiscal accounts. Thanks to this design, it is possible to simulate the model with a wide range of fiscal shocks, which, as shown in the paper, have distinct propagation mechanisms. Having discussed the set-up and the potency of fiscal policy in the model, this paper also includes the following applications for fiscal policy analysis: counterfactual scenarios with alternative fiscal rules, assessment of fiscal policy conducted in the euro area in the past and stochastic fiscal projections.

\subsection*{Paper 2 (with 2 co-authors): How well-behaved are revisions to quarterly fiscal data in the euro area?}
\textbf{Abstract:} Since most macroeconomic data are revised after the initial release both researchers and policy-makers have no choice rather than recognizing and understanding the revisions. This paper analyses revisions to the fiscal data in the euro area, also by contrasting them with the 'better-understood' macro revisions. Concretely, the study verifies whether fiscal revisions fulfil requirements to treat them as well-behaved. To this end, we construct a fiscal quarterly real-time dataset, which contains quarterly releases of Government Finance Statistics and which is supplemented by macro variables from Main National Accounts. Fiscal revisions do not satisfy desirable properties expected from well-behaved revisions. In particular, they tend to have a positive bias, they exhibit a big dispersion, and they are largely predictable. Also, they are similar to macro revisions, in particular since 2014, which contradicts the often heard view about fiscal data being subject to particularly large revisions.

\subsection*{Paper 3 (with 2 co-authors): Evaluating full compliance with fiscal rules: Insights from the old regime for the new EU economic governance framework}
\textbf{Abstract:} After more than two decades of the Stability and Growth Pact (SGP) in operation, it became clear that the framework did not fully achieve its primary objectives. This shortfall can be attributed to two key factors: structural shortcomings of the Pact and poor enforcement. In this paper, we disentangle these two dimensions to assess their impact on the effectiveness of the SGP. We explore whether macroeconomic outcomes in the euro area would have been different had the framework been fully adhered to, even in its imperfect form. Using macroeconomic simulations, we find that strict compliance with the SGP would have significantly altered the euro area's macroeconomic trajectories. By maintaining sounder fiscal positions, as required by the Pact, countries would have been better equipped to undertake counter-cyclical responses during crisis periods. Our analysis underscores the importance of enforcement for the efficacy of any fiscal framework. A key lesson for the newly reformed fiscal rules is that they must be enforced from the outset, as failure to do so risks them failing their envisaged objectives.

\subsection*{Paper 4 (with 2 co-authors): Macroeconomic effects of a large-scale EU investment package: Insights from macro-model simulations}
\textbf{Abstract:} This paper examines the macroeconomic implications of a large-scale EU investment plan aimed at addressing decarbonization, digitalization, and defence challenges. Following the Draghi report's assessment, such transformation requires investment of approximately 5\% of annual EU GDP. Using simulations from two complementary frameworks - the DSGE EAGLE model and the semi-structural ECB-MC model - we analyse the dynamic effects of this investment surge on key macroeconomic variables. Our results suggest a dual impact on real GDP: an initial increase through traditional demand channels, followed by sustained long-term growth driven by productivity enhancements. While inflation rises temporarily in the first years, technological progress eventually brings it back to baseline. The fiscal impact analysis reveals that despite initial strain on public finances, the combination of expanding tax bases and gradual stimulus withdrawal ensures long-term fiscal sustainability. Importantly, the financing structure of the investment package significantly affects budgetary outcomes, with greater private sector participation substantially reducing fiscal costs.

\section*{Expected contribution}
This dissertation is expected to make several contributions to both academic literature and policy practice:

\begin{itemize}[wide=0pt]
    \item \textbf{Theoretical advancement:} Enhance understanding of fiscal policy modelling and transmission mechanisms of fiscal instruments in various models, including those embedded in the MMB database.
    
    \item \textbf{Methodological innovation:} Develop new approaches to analysing fiscal data quality and policy effectiveness through the combination of DSGE and semi-structural modelling.
    
    \item \textbf{Policy implications:} Provide valuable insights for policy implementation by evaluating data quality issues, assessing the importance of fiscal rules, and quantifying the effects of large-scale investment programs.
    
    \item \textbf{Practical tools:} Deliver documented tools and frameworks that will be valuable for policymakers and institutions involved in fiscal policy assessments.
\end{itemize}

\end{document}