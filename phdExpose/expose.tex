\documentclass[11pt,a4paper]{article}

% Required packages
\usepackage[utf8]{inputenc}
\usepackage[T1]{fontenc}
\usepackage{times}
\usepackage[left=2.5cm,right=2.5cm,top=2.5cm,bottom=2.5cm]{geometry}
\usepackage{setspace}
\usepackage{hyperref}
\usepackage{natbib}
\usepackage{amsmath}
\usepackage{booktabs}
\usepackage{enumitem}
\usepackage{xcolor}
\usepackage{titlesec}
\usepackage{fancyhdr}

% Document settings
\onehalfspacing
\setlength{\parindent}{0pt}
\setlength{\parskip}{1em}

% Section spacing
\titlespacing*{\section}{0pt}{3.5ex plus 1ex minus .2ex}{2.3ex plus .2ex}
\titlespacing*{\subsection}{0pt}{3.25ex plus 1ex minus .2ex}{1.5ex plus .2ex}

% Headers and footers
\pagestyle{fancy}
\fancyhead{}
\fancyhead[R]{\thepage}
\fancyhead[L]{Doctoral dissertation exposé}
\fancyfoot{}
\fancyfoot[C]{\thepage}

% Title information
\title{\Large Doctoral dissertation exposé:\\[1em] 
       \huge\textbf{Expectations, uncertainty, and financial decisions:\\
       New insights from ECB survey data}}
\author{Krzysztof Bańkowski\\[0.5em]
        Student ID: 3005 70 007 964\\
        Graduate School of Economics, Finance, and Management (GSEFM)\\
        Goethe University Frankfurt}
\date{\today}

\begin{document}

\maketitle

\begin{abstract}
This dissertation explores expectations formation, uncertainty perception, and financial decision-making in the euro area through four interconnected papers utilizing ECB survey data. The research integrates the Consumer Expectations Survey (CES) and the Survey on Access to Finance of Enterprises (SAFE) to address crucial questions about inflation expectations, uncertainty transmission, household financial resilience, and firm financing constraints. Using methodological approaches including panel data econometrics, machine learning techniques, and information processing models, the dissertation aims to contribute to both academic understanding and practical policy implementation in an era of heightened economic uncertainty.
\end{abstract}

\section*{Research topic and motivation}
This dissertation explores how economic agents form expectations, perceive uncertainty, and make financial decisions in the euro area, with particular focus on the wealth of information contained in the ECB's survey data. The research is motivated by the critical importance of expectations in determining economic outcomes, especially in an environment characterized by multiple overlapping shocks and policy transitions. The timing is particularly relevant as the ECB and other central banks navigate the challenging post-pandemic landscape with persistently high inflation, evolving monetary policy frameworks, and substantial structural changes in European economies. With the relatively recent introduction of the Consumer Expectations Survey (CES) in 2020 and the long-running Survey on Access to Finance of Enterprises (SAFE) since 2009, researchers now have unprecedented granular data to investigate behavioral patterns at both household and firm levels. This research addresses important gaps in our understanding of inflation expectation formation, uncertainty transmission channels, household financial resilience, and small business financing constraints in the euro area.

\section*{Research questions}
\begin{enumerate}[label=\textbf{RQ\arabic*.}, wide=0pt]
    \item How do households form and update their inflation expectations, and what role do socioeconomic characteristics, cognitive abilities, and information channels play in explaining heterogeneity in these expectations?
    \item Through which channels does economic uncertainty affect household consumption and saving decisions, and how can survey-based uncertainty measures improve macroeconomic forecasting?
    \item What determines household financial resilience during economic shocks, and how do financial literacy and access to credit interact with household balance sheet adjustments?
    \item How do firm financing constraints evolve with monetary policy changes, and what are the differential impacts across firm size, sector, and country within the euro area?
\end{enumerate}

\section*{Research objectives}
The dissertation aims to leverage the rich microdata from ECB surveys to enhance our understanding of expectation formation and financial decision-making in the euro area. Specifically, it seeks to:
\begin{itemize}[wide=0pt]
    \item Identify the key determinants of household inflation expectations and quantify the role of information frictions and cognitive limitations
    \item Develop novel survey-based measures of uncertainty and establish their relationship with consumption and saving behavior
    \item Analyze the dynamics of household financial vulnerability and resilience during economic downturns
    \item Assess the transmission of monetary policy through firm financing conditions and document heterogeneities across the business landscape
\end{itemize}
Through this multi-faceted approach, the research strives to bridge the gap between theoretical models of expectation formation and observed behavior, with implications for monetary policy effectiveness and financial stability.

\section*{Dissertation papers}

\subsection*{Paper 1 (single-authored): The anatomy of inflation expectations: New evidence from the ECB Consumer Expectations Survey}
\textbf{Abstract:} This paper investigates the formation and updating of inflation expectations using granular data from the ECB's Consumer Expectations Survey. Leveraging the panel structure of the data and the survey's rich set of socioeconomic variables, I document substantial heterogeneity in inflation expectations across demographic groups, income levels, and educational attainment. I find that information acquisition patterns significantly explain this heterogeneity, with a notable "gender expectation gap" that persists even after controlling for financial literacy and news consumption. The paper introduces a novel approach to measuring expectation updating by exploiting the survey's monthly frequency and randomized information treatments. Results indicate that households with higher cognitive abilities and financial literacy display greater sensitivity to official inflation statistics and central bank communications. Additionally, I document evidence of "expectation spillovers" where geopolitical uncertainty and energy price shocks influence expectations about unrelated economic variables. These findings have important implications for monetary policy communication strategies and the anchoring of inflation expectations during volatile economic periods.

\subsection*{Paper 2 (with 2 co-authors): Uncertainty transmission channels: Linking survey evidence to consumption dynamics in the euro area}
\textbf{Abstract:} This paper examines how perceptions of economic uncertainty influence household consumption decisions in the euro area. We construct multidimensional household-level uncertainty indicators from the ECB's Consumer Expectations Survey, distinguishing between macroeconomic uncertainty (concerning inflation, unemployment, and economic growth) and household-specific uncertainty (concerning income, employment, and housing). Using a panel vector autoregression approach, we identify distinct transmission channels through which uncertainty affects consumption: the precautionary savings channel, the postponement of durables channel, and the income expectation channel. Our findings reveal substantial heterogeneity in uncertainty sensitivity across consumption categories and household characteristics. The postponement effect is particularly strong for households with liquidity constraints, while the precautionary channel dominates for wealthier households. We also demonstrate that incorporating our survey-based uncertainty measures significantly improves forecasting models for aggregate consumption, outperforming traditional volatility-based uncertainty proxies. The results suggest that targeted fiscal policy measures addressing specific uncertainty channels could enhance effectiveness during periods of heightened economic insecurity.

\subsection*{Paper 3 (with 2 co-authors): The determinants of household financial resilience: Evidence from the ECB Consumer Expectations Survey}
\textbf{Abstract:} This paper investigates the factors that determine household financial resilience during economic downturns, using the ECB's Consumer Expectations Survey data collected through the pandemic period and subsequent inflation surge. We define and measure financial resilience through a multidimensional index capturing households' ability to weather income shocks without significant consumption adjustments or financial distress. Our analysis reveals that while income and wealth levels predict baseline resilience, the dynamic adjustments to shocks vary considerably across seemingly similar households. We identify four key determinants of superior resilience: financial literacy, access to liquid assets, diversified income sources, and the availability of formal credit lines. Particularly notable is our finding that financial knowledge exhibits complementarity with formal credit access—financially literate households make more effective use of credit facilities during stress periods. Using a difference-in-differences approach with the introduction of targeted policy support measures as exogenous variation, we demonstrate that government interventions are most effective for households with moderate financial literacy but limited liquidity. These results provide valuable insights for designing more targeted financial stability policies and developing early warning indicators of household financial vulnerability.

\subsection*{Paper 4 (with 2 co-authors): Monetary policy transmission through SME financing conditions: Insights from the Survey on Access to Finance of Enterprises}
\textbf{Abstract:} This paper examines how monetary policy changes affect small and medium-sized enterprises' access to finance in the euro area, using comprehensive data from the ECB's Survey on Access to Finance of Enterprises (SAFE). Exploiting the survey's semi-annual frequency and its coverage of multiple monetary policy cycles since 2009, we investigate the transmission channels through which policy rate changes influence SME financing conditions. Our identification strategy leverages the differential timing of survey waves across countries relative to monetary policy announcements. We document three key findings: First, there is substantial heterogeneity in monetary policy pass-through across firm sizes, with micro enterprises experiencing both delayed and amplified reactions compared to medium-sized firms. Second, we identify significant country-level asymmetries in transmission that cannot be explained by banking sector concentration alone, suggesting important interaction effects with national institutional frameworks. Third, using text analysis of open-ended survey responses, we uncover a "confidence channel" through which monetary policy communication affects firms' perception of future financing conditions independently of actual credit supply changes. These findings have important implications for calibrating monetary policy in the heterogeneous currency union and for designing complementary policies to address financing gaps for vulnerable enterprise segments.

\section*{Expected contribution}
This dissertation is expected to make several contributions to both academic literature and policy practice:

\begin{itemize}[wide=0pt]
    \item \textbf{Theoretical advancement:} Enhance understanding of expectation formation processes by incorporating insights from behavioral economics and information frictions into traditional macroeconomic models.
    
    \item \textbf{Methodological innovation:} Develop new approaches to measuring and analyzing uncertainty using survey microdata, including techniques to address measurement error and reporting biases in expectation surveys.
    
    \item \textbf{Policy implications:} Provide valuable insights for monetary policy communication strategies, macroprudential policy design, and targeted interventions to address household financial vulnerability and SME financing constraints.
    
    \item \textbf{Practical tools:} Deliver documented indicators and early warning systems that will be valuable for policymakers and institutions in monitoring household financial resilience and firm financing conditions.
\end{itemize}

\end{document}